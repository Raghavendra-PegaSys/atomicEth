\documentclass{article}
\usepackage[hyphens]{url}
\usepackage{graphicx}

\title{Algorithm for Cross-shard Cross-EE Atomic User-level \eth Transfer in Ethereum 2}
\author{Raghavendra Ramesh\\Consensys Software R \& D\\raghavendra.ramesh@consensys.net}
\date{}

\newcommand{\eth}[0]{ETH~}
\newcommand{\tocredit}[0]{{\bf ToCredit}}

\begin{document}

\maketitle

\section*{Abstract}
We address the problem of atomic cross shard value transfer in Ethereum 2. We leverage on Ethereum 2 architecture, more specificially on Beacon chain and crosslinks, and propose a solution on top of the netted-balance approach that was proposed for EE-level atomic \eth transfers. We split a cross-shard transfer into two transactions: a debit and a credit. First, the debit transaction is processed at the source shard. The corresponding credit transaction is processed at the destination shard in a subsequent block. We use {\em netted} shard states as channels to communicate pending credits and pending reverts. We discuss various scenarios of debit failures and credit failures, and show our approach ensures atomicity. 

The benefits of our approach are that we do not use any locks nor impose any constraints on the Block Proposer to select specific transactions. However we inherit the limitation of an expensive operation from the netted-balance approach of querying partial states from all other shards.

\section{Introduction}

Ethereum 2 is planned to have 64 shards. The blocks on the shard chains synchronise with the Beacon chain using crosslinks. The crosslinks are the stateroots (root of Merkle tree of the shard state), and provide a means to validitate the integrity of a shard state on another shard. 

The execution in Ethereum 1 happens in a fixed fashion using EVM and fixed notion of account id and balances. The plan is to generalise this notion of execution environment (EE) in Ethereum 2. For instance, an environment similar to Bitcoin can be hosted on Ethereum 2 as one EE.  Some EEs could use another virtual machine instead of EVM. The EEs and users of them are distributed across multiple shards. In such a scenario, the cross-shard cross EE \eth transfers become necessary. This paper presents an algorithm for atomic user-level \eth transfers between two EEs hosted on two different shards.

Because one execution environment is complete in its own means, we need to maintain \eth balances of each of these execution environments separately. An EE could provide another currency on top of \eth. Users are associated with EEs in the sense that an EE serves as a home for multiple users. The problem of transferring value from a user $a$ of EE $E_1$ to another user $b$ of another EE $E_2$ involves two aspects:
\begin{enumerate}
    \item the transfer of \eth from EE $E_1$ to EE $E_2$, and
    \item the transfer of values from the user $a$ to user $b$.
\end{enumerate}

Vitalik Buterin proposed a {\em netted balance approach} (\cite{netted-balance}) for atomic cross-shard transfer of \eth between EEs. This paper proposes user-level atomic cross-shard transfer on top of this netted balance approach of EE-level transfers. We have published this approach at~\cite{ethres-raghavendra}.

\section{Netted balance approach}
\label{sec:netted-balance}
Vitalik Buterin proposed a netted balance approach (~\cite{netted-balance}) for atomic cross-shard transfer of \eth between EEs. The naïve idea is for every shard to maintain its balance of every EE. In the netted-balance approach, every shard maintains the part-balances of every EE on every other shard. 

Suppose we have three shards $A$, $B$ and $C$. Consider an EE $e$. Then, the balance information of $e$ on shard $A$ is distributed on all shards $A$, $B$ and $C$. So, each shard now stores an ordered triple of part-balances. For example the $e$'s balance information on $A$ is stored as an ordered triple of the form $(10, 20, 30)$, meaning $e$'s part-balance on shard $A$ is 10, $e$'s part-balance on shard $B$ is 20, and $e$'s part-balance on shard $C$ is 30. Continuing the example, suppose we have triples $(-5, 10, 20)$ on $B$ and $(1,-2,3)$ on $C$. To obtain $e$'s full balance on shard $A$, we need to sum up $e$'s part balances on shards $A$, $B$, and $C$, i.e., $10 + (-5) + 1 = 6$ \eth. 

The benefit of this arrangement is that the transfer of $x$ \eth from shard $A$ to shard $B$ can be affected by an intra-shard operation only on the source shard (operation completely on $A$ here) by subtracting $x$ \eth from $A$'s part-balance and adding $x$ to $B$'s part-balance. The $A$'s triple changes from $(10, 20, 30)$ to $(10-x, 20+x, 30)$. The triples on other shards are not touched. Thus, the cross-shard EE-level \eth transfer is accomplished by a single atomic operation on a single shard.

However, the downside is that querying of an EE's balance on a shard is not a single operation because all the part balances on all other shards need to be fetched and summed. 

This idea naturally extends to cross EE transfers too.

\section{Atomic User-level Transfer}
\label{sec:atomic-user}
The core idea of this proposal is to extend and leverage the netted-balance approach of distribution of EE-balances to outstanding user-level credits and outstanding user-level reverts. The netted state (extends the netted balance) is used as a channel to communicate outstanding credits to recipient shard and outstanding reverts to the sender shard. We now describe the approach in full detail now.

\subsection{Preliminaries}

For $i$ in natural numbers, let
$s_i$'s denote shards, 
$E_i$'s denote execution environments,
$a_i, b_i$'s denote users.
We use the concept of {\em System Event messages} from~\cite{peter-cross-shard}. These messages are similar to application event messages in contract code, but are unforgeable by the application. 

In our approach, we use one System Event message called \tocredit, which includes:
\begin{itemize}
\item sender details (shard-id, EE-id, user address), 
\item recipient details (shard-id, EE-id, user address), 
\item transfer amount, 
\item the block number of the shard block where this event is emitted, and 
\item an index number starting from 0 (for every block).
\end{itemize}
We use \tocredit($a,x,b$) to denote a system event with sender details of user $a$, the transfer amount $x$, and the receipient details of the user $b$, and elide the block number and the index number when they are obvious from the context.

\subsection{Transactions}

A cross-shard transfer of $x$ \eth from an user $a_i$ on $(s_1,E_1)$ to an user $b_i$ on $(s_2,E_2)$ is split by our system into two transactions in a natural way: a cross-shard debit transfer ($\Longrightarrow$) and a cross-shard credit transfer ($\longrightarrow$), corresponding to deducting $x$ on the source side and adding $x$ to the destination side respectively. 

{\bf Cross-shard debit transfer} transaction is signed by the sender $a_i$, and the signature is stored in the fields $v, r$ and $s$ as in Ethereum 1. It contains a unique transaction identifier. It is submitted on sender shard, and emits a \tocredit System event on success.

{\bf Cross-shard credit transfer} transaction is submitted on recipient shard, and includes the \tocredit System Event and the Merkle Proof for it.

\subsection{Shard State}
In the netted-balance approach, each shard $s$ stores the part-balances of every EE on every shard. It can then be seen as a matrix, say $s.partBalance$ of size number of shards $\times$ number of EE's. Here, $s.partBalance[s_i,E_j]$ gives the part-balance of EE $E_j$ on shard $s_i$, which is recorded at shard $s$. The real balance of EE $E_j$ on shard $s_i$ is given by : 
\[
	realBalance[s_i,E_j] = \sum_k s_k.partBalance[s_i,E_j].
\]

We introduce more elements on top of part balances in the shard state, and we use the name $partState$ matrix in place of $partBalance$ matrix. Each cell of this matrix has 3 elements, as shown in Figure~\ref{fig:shardstate}. They are:
\begin{itemize}
	\item $partBalance$ (as in netted-balance approach),
	\item $credits$, the set of cross-shard credit transactions that needs to be imported on the destination shard,
	\item $reverts$, the set of cross-shard revert transactions that needs to be effected.
\end{itemize}

\begin{figure}[h]
	\includegraphics[scale=0.5]{state.jpg}
	\caption{Shard State\label{fig:shardstate}}
\end{figure}
	
The shard state contains a map called $outstandingCredits$. At the time of processing a block $k$ on shard $s$, for every EE $E$, $s.outstandingCredits$ adds the outstanding credit transactions targeted to $(s,E)$ from a shard $s'$ and an EE $E'$ whose corresponding debit transactions were successfully processed in the immediately previous block at shard $s'$. So, we have:
\[
	s.outstandingCredits[E][s', E', k-1] = \{~e ~|~ e \in s'.partState[s,E].credits\}
\]
The $k-1$ in the equation indicates that the corresponding debit transactions were processed at $k-1$ block. 

There are two ways to remove an entry from $s.outstandingCredits$. One, is by the successful proccessing of the credit transfer transaction, the other is when we {\em time-out} processing of the credit transfer. The time-out is required to guarantee a bound on the transfer time. Typically time-outs are specified in terms of number of blocks.

For every EE $E$, the shard state contains the $EETransferAmount$ that bookkeeps the net amount to be transferred at EE-level.

For every user of every EE, the shard state also stores the user's balance. We use $s.userBalance[E,a]$ to denote the balance of user $a$ of EE $E$ on shard $s$. Because the primary purpose of the shards is to divide the content, the users' balances are not distributed. Users are typically assigned a shard.

\subsection{Block Proposer Algorithm}
\label{sec:BP}
We now present the main contribution: the algorithm for the block proposer in handling cross-shard cross-EE transfers with the guarantee of atomicity. As previously mentioned, the idea is to split a cross-shard transfer into a debit and a credit. First, a debit transaction is processed at the source shard, and the EE-level transfer is fully effected. If successful, the corresponding user-level credit transaction is queued on to the destination shard, which is processed in a subsequent block. If the credit fails, then we do the EE-level revert and user-level reverts NOT as separate transactions but as enshrined execution processing. 

Without loss of generality, assume that a Block Proposer (BP) is proposing a block numbered $k$ on shard $s_1$. Then BP executes the following steps for every EE $E_i$. 

\begin{enumerate}
\item {\bf Initialisation.} Obtain the part states of $E_i$ from every shard. Ensure that the obtained $s_i.partState$s,  $1 \le i \le n$, are correct using Merkle Proofs and crosslinks.
Compute the real balance of $E_i$ on $s_1$ using 
\[
	realBalance[s_1,E_i] = \sum_n s_n.partState[s_1,E_i].balance.
\]
For every shard $s$ and every EE $E$, set $s_1.EETransferAmount[s,E]$ to $0$.

\item {\bf Preprocess pending credits}
    \begin{itemize}
        \item Add entries to $s_1.outstandingCredits$
        $[s', E', (k-1)] \mapsto \bigcup_n s_n.partState[s_1,E_i].credits$
        \item Kick out expired credits from $s_1.outstandingCredits$. If there is an entry with $[s',E',k']$ such that $k' + timeOut == k$ then do:
        \begin{itemize}
            \item $s_1.partState[s',E'].reverts := s.outstandingCredits[s', E', k']$.
            \item Delete the entry with $[s', E', k']$.
			\item $s_1.EETransferAmount[s',E'] := \sum_r x_r$ where \\
			$r \in s_1.partState[s',E'].reverts$ and $x_r$ denotes $r$'s transfer amount.
        \end{itemize}
	\end{itemize}
	
\item {\bf Process user-level reverts}. For every $r \in s_1.partState[s_1,E_i].reverts$, $s_1.userBalance[E_i,sender(r)] += x_r$, where $x_r$ denotes $r$'s transfer amount. After this operation, $s_1.partState[s_1,E_i].reverts$ is set to $\emptyset$.

% \item {\bf Reset.} For every pair $(s_2,E_j)$, reset the following.
% \begin{itemize}
% 	\item $s_1.partState[s_2,E_j].credits = \emptyset$
% 	\item $s_1.partState[s_2,E_j].reverts = \emptyset$
% \end{itemize}

\item {\bf Process transactions.} For every pair $(s_2,E_j)$
    \begin{enumerate}
	\item Select cross-shard transactions $t_1,\ldots,t_m$ between $(s_1,E_i)$ and $(s_2,E_j)$ to be included in the block. It can be a new transaction from the transaction pool, or a credit transaction from $s_1.outstandingCredits[s_2,E_j]$. 
	
	\item For every $n \in \{1, \ldots, m\}$: 
		\begin{itemize}
		\item If $t_n$ is a cross-shard debit transaction of the form $a_n \stackrel{x_n}{\Longrightarrow} b_n$ and $realBalance(s_1,E_i) > s_1.EETransferAmount(s_2,E_j) + x_n$
		\begin{itemize}
			\item include $t_n$ to the block,
			\item if $t_n$ executes successfully 
			\begin{itemize}
				\item $s_1.userBalance(E_i,a_n)$ -= $x_n$ (implied with successful execution of $t_n$)
				\item $s_1.EETransferAmount[s_2,E_j] += x_n$
				\item emit \tocredit($a_n, x_n, b_n$) System Event
				\item $s_1.partState[s_2,E_j].credits ~ \cup= ~ \{a_n \stackrel{x_n}{\longrightarrow} b_n\}$
			\end{itemize}
		\end{itemize}
		\item Else if $t_i$ is a cross-shard credit transaction of the form $b_n \stackrel{x_n}{\longrightarrow} a_n$ AND Merkle Proof check of the included \tocredit~ System Event passes AND $realBalance(s_1,E_i) > s1.EETransferAmount(s_2,E_j) + x_n$
		\begin{itemize}
			\item include $t_n$ to the block
			\item Remove $t_n$ from $s_1.outstandingCredits[s_2,E_j,k']$ where the block number $k'$ is derived from the included \tocredit ~System Event.
			\item if $t_n$ executes successfully then $s_1.usersBalance[E_i,a_n] += x_n$ (implied by the successful execution of $t_n$)
			\item if it fails \\
			$s_1.partState[s_2,E_j].reverts ~\cup=~ \{(b_n,x_n,a_n)\}$\\
			$s_1.EETransferAmount[s_2,E_j] += x_n$.
		\end{itemize}
	\end{itemize}
	\item Process EE-level transfer\\
		$s_1.partState[s_1,E_i].balance -= s1.EETransferAmount[s_2,E_j]$, \\
		$s_1.partState[s_2,E_j].balance += s1.EETransferAmount[s_2,E_j]$.
    \end{enumerate}
\end{enumerate}


\subsection{Features of the algorithm}
Let us visit some of the notable features of the above algorithm.
\begin{enumerate}
\item Before processing a pending cross-shard credit transfer transaction $b_i \stackrel{x_i}{\longrightarrow} a_i$, the EE-level transfer is already complete. 
\item User-level reverts happen in the immediate next slot after a failed or an expired credit transfer. The EE-level revert happens in the same slot as the failed / expired credit transfer. This technique pushes the revert to the EE host functions instead of treating them as separate transactions. This avoids complex issues like revert timeouts and revert gas pricing.
\item Transaction identifiers need to be unique inside the time-out blocks window.
\item There is a corner case where the sender disappears by the time revert happens, then we end up in a state where there is \eth loss at user-level, but not at EE-level. We feel this is the best state to be in when such a situation happens.
\item A transaction is not included in a block if the EE does not have sufficient balance. Note that EE balance check is done even for credit transfers because of potential reverts.
\item A time-out is required to kick out long pending credit transfers. The second bullet of step 3 describes this procedure. The idea is to move out the expired user-level credit transfers, convert them into user-level reverts on the sender shard, thus achieving a fixed size for $outstandingCredits$ datastructure.
\item No locking / blocking.
\item No constraint on the block proposer to pick specific transactions or to order them.
\item {\bf Main goal.} Atomicity of a cross-shard transfer.
\item {\bf Demerit.} In every block, a BP has to get the outstanding credits and reverts from every other shard. This is inherited from the netted balance approach, where a BP requires the netted balances from all shards. However, in the EE-level netted-balance approach the querying is restricted to the sender EE's that are derived from the user-level transactions included in the block. The problem is aggravated here, because we need to query from all EE's, even for the EE's not touched in this block.
\end{enumerate}

% BitFieldMap
% The datastructure *BitFieldMap* is stored in the shard state to protect against replays of credit transfers and to time-out credit transfers. It is a map from a block number to a set of elements of the form $(t_i \mapsto b_i)$, where the block number identifies the block on the sender shard where a cross-shard debit transfer $t_i$ happened, and $b_i$ is a bit initially set to 0.  
% - *BitCheck*($t_i$) returns true when there is $(t_i \mapsto 0)$ for any block number. Because transaction identifiers are unique, a transaction appears only once in *BitFieldMap*. Zero indicates that the credit is not processed yet. One indicates that the credit is already processed. If no entry is present, then this is an invalid credit and false is returned.
% - *SetBit*($t_i$) sets the bit for $t_i$ to 1 for the block number that is already existing. 

% A *timeBound* is required to kick out long pending credit transfers. The second bullet of step 2 describes this procedure. The idea is to move the expired user-level credit transfers as user-level reverts on the sender shard, and achieves a fixed size for *BitFieldMap*.


% Processing Revert Transfers
% A revert transfer is placed in the shared portion of the EE-level state when either the corresponding credit transfer fails or expires. It is stored in the local shard state, but in the portion that is reserved for the sender shard. The step 4 collects all the portions and processes the user-level reverts. 

% Note that the EE-level revert happens on the recipient shard in the same block where the credit transfer fails or expires. However, the user-level revert happens in the very next slot on the sender shard. This technique pushes the revert to the EE host functions instead of treating them as separate transactions. This avoids complex issues like revert timeouts and revert gas pricing.

\section{Scenarios}
In this section, we apply the algorithm from Section~\ref{sec:BP} in different kinds of representative scenarios and show how atomicity is preserved.

Assume that $a_1, a_2, a_3$ are users on $(s_1,E_1)$ and $b_1,b_2,b_3$ are users on $(s_2,E_2)$.

\subsection{Optimistic case}
Let us look at the optimistic case, where everything happens as expected. Suppose we include three cross-shard transactions as shown in Figure~\ref{fig:optimistic} in a block on the shard $s_1$. In the very next block on $s_2$, the $s_2.outstandingCredits$ is updated with 3 pending credit transactions. In the same block or subsequently in some block on $s_2$ these credit transactions are processed. 

Note that the EE-level transfers are complete at the $s_1$'s block itself. The $s_2$'s block could process the credit transactions all in the same block or in different blocks. The blocks processing the credit transfers could include new cross-shard transactions as shown in Figrue~\ref{fig:optimistic} or other intra-shard transactions.

\begin{figure}[h]
	\includegraphics[scale=0.4]{OptmisiticCase.jpg}
	\caption{Optimistic case\label{fig:optimistic}}
\end{figure}

\subsection{When credit fails}
Let us look at the case when the processing of a credit transaction fails. Consider a slight variation from the above scenario, where the credit transaction $a_3 \stackrel{30}{\longrightarrow} b_3$ fails or is expired. Then in the same block the EE-level revert happens (with $E_1$ getting its $30$ \eth back), and finally in the very next block on $s_1$ the user-level reverts are processed.

\begin{figure}[h]
	\includegraphics[scale=0.4]{Credit.jpg}
	\caption{When credit fails\label{fig:credit}}
\end{figure}

\subsection{When debit fails}
Consider the case when the initial debit transaction fails. This is a very simple case, as nothing needs to be done. Simply this transaction has to be resubmitted as in the case of Ethereum 1.

\begin{figure}[h]
	\includegraphics[scale=0.4]{Debit.jpg}
	\caption{When debit fails\label{fig:debit}}
\end{figure}

\section{Threat Analysis of a Byzantine Block Proposer}
\label{sec:threat}
Talking with Roberto Saltini (PegaSys, ConsenSys) @saltiniroberto revealed many scenarios, especially the case when a Block Proposer (BP) is Byzantine. A Byzantine Block Proposer (BBP) might choose to deviate from the above algorithm. It becomes clear from the following that the protocol withstands such a BBP.

Checks done by a Validator / Attester
Verify that the received part states from other shards for all EEs are correct.
Verify that the data structure BitFieldMap is populated with the impending credits for this shard.
Verify that the impending reverts are processed, meaning the sender users are credited with the transfer amount.
Verify that correct ToCredit System Events are emitted for included and successful cross-shard debit transfer transactions.
Verify that correct outgoing credit transfers are written to the appropriate shared state.
Verify that the corresponding bit is set when a cross-shard credit transfer happens successfully.
Verify that a correct revert transfer is placed for a failed cross-shard credit transfer transaction.
Check that the correct amount is transferred at the EE-level.
Because a validator / an attester has access to the current shard state, (s)he can verify points: 2, 3, 5, 6, 7, 8. An attester is also given with the partStates from other shards along with their Merkle Proofs and (s)he has access to crosslinks form the Beacon block. So, (s)he can check points 1 and 8, that is, verify part balances, impending credits and impending reverts. Also because (s)he has access to all the transaction receipts of the transactions included in the block, (s)he can check point 4.

So, if a BBP chooses to

show no or false
part EE-balances or
set of impending credits or
set of reverts, or
not update or wrongly update BitFieldMap with impending credits, or
not process or wrongly process impending reverts, or
not emit or emit with incorrect data the ToCredit System Event
not $SetBit(t_i)$, or
not include a revert for a failed credit transaction, or
not affect appropriate EE-level transfer,
his / her block will be invalidated by the attesters, assuming that the two-thirds of the attesters are honest.

Also, in the discussion, @saltiniroberto pointed out that BitFieldMap is more of a set rather than a map to bits. It can be replaced by a set, where BitCheck checks membership and SetBit removes the element from the set.

\section{Conclusion}
Future work
* Optimise the space requirement for storing outstanding credits and outstanding reverts.
* Explore caching for optimising the reads of partStates of every EE of every other shard in every block (related to the above mentioned demerit).

\section*{Acknowledgements}
We thank Roberto Saltini, Peter Robinson, Nicholas Liochon and David Hyland-Wood for all the insightful discussions.

\bibliographystyle{plain}
\bibliography{refs}
\end{document}